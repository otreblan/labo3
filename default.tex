\documentclass[12pt]{article}

\usepackage[pdfa, pdfusetitle, unicode=true]{hyperref}
\usepackage[spanish]{babel}
\usepackage[margin=1.5cm, a4paper]{geometry}
\usepackage{luacode}
\usepackage[shortlabels]{enumitem}
\usepackage{import}
\usepackage{xcolor}
\usepackage{ifluatex}

\usepackage{silence}
\WarningFilter{latex}{You have requested package}
\usepackage{luapackageloader}

%Esto sirve para poner ecuaciones
\usepackage{mathtools}
\usepackage{cancel}

%Esto sirve para poner imágenes {{{
\usepackage{graphicx}
\usepackage{svg}
\usepackage{subcaption}

\usepackage{float}
\usepackage{pgfplots}
\usetikzlibrary{spy, backgrounds}

\pgfplotsset{compat=1.16}
\graphicspath{ {ima/} }
%}}}

%Colores de los links {{{
\definecolor{red}{HTML}{F22C40}
\definecolor{green}{HTML}{5AB738}
\definecolor{yellow}{HTML}{D5911A}
\definecolor{blue}{HTML}{407EE7}
\definecolor{magenta}{HTML}{6666EA}
\definecolor{cyan}{HTML}{00AD9C}

\hypersetup{
	colorlinks=true,
	linkcolor=blue,
	urlcolor=cyan,
}
%}}}

%Esto controla a la cabecera {{{
\usepackage{fancyhdr}

\pagestyle{fancy}
\fancyhf{}
\renewcommand{\headrulewidth}{0pt}
\chead{ \textbf{\normalsize{Física I} }}
\fancyhf[HL]{\includesvg[height=0.8\headheight]{Utec.svg}}
\fancyfoot[C]{\textbf{\thepage}}
\setlength{\headheight}{52pt}
\setlength{\textheight}{700pt}
%}}}

\title{\textbf{Laboratorio}}
%Aqui hay que poner a los autores
\author{
		Alberto Oporto Ames\\
		\texttt{alberto.oporto@utec.edu.pe}
		}

%Aquí empieza el documento
\begin{document}
\maketitle
\thispagestyle{fancy}

\section{Dcl's}%
\label{sub:dcl}
\begin{figure}[H]
	\centering
	\includesvg[width=\linewidth]{dcls.svg}
\end{figure}

\section{Preguntas}%
\subsection{Utilizando DINÁMICA, determinar teóricamente lo que debería medir el sensor.
Respecto al tiempo, ¿qué tipo de curvas son? Explicar.}%
\begin{align*}
	m_p &= 0.1065 kg && \text{Masa del disco de inercia}\\
	m_1 &= 1.0661 kg && \text{Masa del carrito}\\
	m_2 &= 0.1 kg && \text{Masa en el porta pesas}\\
	r_1 &= 0.024 m && \text{Radio 1 de la polea}\\
	r_2 &= 0.0145 m && \text{Radio 2 de la polea}\\
	g &= 9.81 \frac{m}{s^2}  && \text{Gravedad}
\end{align*}
\begin{align*}
	\frac{\alpha(2m_pr_2^2+m_1r_1^2)}{4r_2(g-r_2\alpha)} &= m_2\\
	\alpha &= \frac{4r_2m_2g-4r_2^2m_2\alpha}{2m_pr_2^2+m_1r_1^2}\\
	\alpha + \frac{4r_2^2m_2\alpha}{2m_pr_2^2+m_1r_1^2} &= \frac{4r_2m_2g}{2m_pr_2^2+m_1r_1^2}\\
	\alpha \Big{(}1 + \frac{4r_2^2m_2}{2m_pr_2^2+m_1r_1^2}\Big{)} &= \frac{4r_2m_2g}{2m_pr_2^2+m_1r_1^2}\\
	\alpha &= \frac{4r_2m_2g}{(2m_pr_2^2+m_1r_1^2)\Big{(}1 + \frac{4r_2^2m_2}{2m_pr_2^2+m_1r_1^2}\Big{)}}\\
	\alpha &= 76.58 \frac{rad}{s^2}
\end{align*}
Constante.
Porque no depende del tiempo.

\subsection{A partir de la medición realizada, ¿se observan las curvas esperadas?}%

%Sin esta cosa pgfplots da errores. {{{
\shorthandoff{"}

%https://tex.stackexchange.com/questions/340661/argument-of-languageactivearg-has-an-extra-i-use-includegraphics-and-r/340663#340663

%d=30cm{{{
\begin{figure}[H]
	\centering
	\begin{tikzpicture}
		% Opciones {{{
		[
			spy using outlines={rectangle,
				magnification=2,
				width=7cm,
				height=3cm,
				connect spies
			}
		]%}}}
		\begin{axis}[
			% Opciones {{{
			width=0.9\linewidth,
			height=6cm,
			title=\textbf{$\alpha_1,\ d=30cm$},
			axis y line=left,
			axis x line=bottom,
			axis line style = ultra thick,
			xlabel={Tiempo ($s$)},
			ylabel={Aceleración angular ($\frac{rad}{s^2}$)},
			ymajorgrids=true
			]%}}}
			\addplot[
				% Opciones {{{
				restrict x to domain=0:4,
				draw opacity=0.1,
				color=red,
				smooth,
				ultra thick
				]%}}}
				table
				[
					col sep=comma,
					y index=3,
					x index=0
				]
				{d1.csv};
			\addplot[
				% Opciones {{{
				restrict x to domain=0.5:1.3,
				color=red,
				smooth,
				ultra thick
				]%}}}
				table
				[
					col sep=comma,
					y index=3,
					x index=0
				]
				{d1.csv};
			\coordinate (ripple) at (0.87,-50);
			\coordinate (lupa) at (2.75,100);
		\end{axis}
		\spy[black] on (ripple) in node[fill=white] at (lupa);
	\end{tikzpicture}
\end{figure}%}}}

%zad=20cm{{{
\begin{figure}[H]
	\centering
	\begin{tikzpicture}
		% Opciones {{{
		[
			spy using outlines={rectangle,
				magnification=2,
				width=7cm,
				height=3cm,
				connect spies
			}
		]%}}}
		\begin{axis}[
			% Opciones {{{
			width=0.9\linewidth,
			height=6cm,
			title=\textbf{$\alpha_2,\ d=20cm$},
			axis y line=left,
			axis x line=bottom,
			axis line style = ultra thick,
			xlabel={Tiempo ($s$)},
			ylabel={Aceleración angular ($\frac{rad}{s^2}$)},
			ymajorgrids=true
			]%}}}
			\addplot[
				% Opciones {{{
				restrict x to domain=0:4,
				draw opacity=0.1,
				color=red,
				smooth,
				ultra thick
				]%}}}
				table
				[
					col sep=comma,
					y index=3,
					x index=0
				]
				{d2.csv};
			\addplot[
				% Opciones {{{
				restrict x to domain=0.6:1.15,
				color=red,
				smooth,
				ultra thick
				]%}}}
				table
				[
					col sep=comma,
					y index=3,
					x index=0
				]
				{d2.csv};
			\coordinate (ripple) at (0.87,-50);
			\coordinate (lupa) at (2.75,100);
		\end{axis}
		\spy[black] on (ripple) in node[fill=white] at (lupa);
	\end{tikzpicture}
\end{figure}%}}}

%d=40cm{{{
\begin{figure}[H]
	\centering
	\begin{tikzpicture}
		% Opciones {{{
		[
			spy using outlines={rectangle,
				magnification=1.8,
				width=7.5cm,
				height=3cm,
				connect spies
			}
		]%}}}
		\begin{axis}[
			% Opciones {{{
			width=0.9\linewidth,
			height=6cm,
			title=\textbf{$\alpha_3,\ d=40cm$},
			axis y line=left,
			axis x line=bottom,
			axis line style = ultra thick,
			xlabel={Tiempo ($s$)},
			ylabel={Aceleración angular ($\frac{rad}{s^2}$)},
			ymajorgrids=true
			]%}}}
			\addplot[
				% Opciones {{{
				restrict x to domain=0:4,
				draw opacity=0.1,
				color=red,
				smooth,
				ultra thick
				]%}}}
				table
				[
					col sep=comma,
					y index=3,
					x index=0
				]
				{d3.csv};
			\addplot[
				% Opciones {{{
				restrict x to domain=0.7:1.65,
				color=red,
				smooth,
				ultra thick
				]%}}}
				table
				[
					col sep=comma,
					y index=3,
					x index=0
				]
				{d3.csv};
			\coordinate (ripple) at (1.20,-100);
			\coordinate (lupa) at (3,100);
		\end{axis}
		\spy[black] on (ripple) in node[fill=white] at (lupa);
	\end{tikzpicture}
\end{figure}%}}}
\shorthandon{"}%}}}

Casi.
No son completamente rectas y además el sistema estaba al revés.

\subsection{Comparar en valor de la aceleración promedio medida con el teórico. Comentar.}%

\directlua{
	package.path='/usr/share/lua/5.3/?.lua'
	csv = require('csv')
}

\directlua{
	function promedio(file, min, max)
	local f = csv.open(file)

	local sum = 0
	local counter = 0
	local datos = {}

	for fields in f:lines() do
		local time = tonumber(fields[1])
		if time then
			if ( time >= min and time <= max) then
				counter = counter + 1
				sum = sum + fields[4]
				datos[counter] = fields[4]
			end
		end
	end
	return{(sum/counter), counter, datos}
	end
}

\directlua{
	promedios = {}
	promedios[1] = promedio('d1.csv', 0.5, 1.3)
	promedios[2] = promedio('d2.csv', 0.6, 1.15)
	promedios[3] = promedio('d3.csv', 0.7, 1.65)

	promedio = (promedios[1][1]+promedios[2][1]+promedios[3][1])/3

	samples = promedios[1][2]+promedios[2][2]+promedios[3][2]
}

Valor medio de la aceleración angular:
\begin{align*}
	\overline{\alpha} &= \frac{\overline{\alpha_1}+\overline{\alpha_2}+\overline{\alpha}_3}{3}  \\
	\overline{\alpha}&=\frac{
		\directlua{tex.print(promedios[1][1])}
		\directlua{tex.print(promedios[2][1])}
		\directlua{tex.print(promedios[3][1])}}{3}
	\frac{rad}{s^2} = \directlua{tex.print(promedio)} \frac{rad}{s^2}
\end{align*}

Valor teórico:
\begin{align*}
	\alpha =76.58 \frac{rad}{s^2}
\end{align*}

Observaciones tomadas $n_n$
\begin{align*}
	n &= n_1 + n_2 + n_3\\
	n &= \directlua{tex.print(promedios[1][2])} +\directlua{tex.print(promedios[2][2])}+\directlua{tex.print(promedios[3][2])}\\
	n &= \directlua{tex.print(samples)}
\end{align*}

\directlua{
	varianIncompleta = 0
	for ii=1, 3 do
		for jj=1, promedios[1][2] do
			varianIncompleta = varianIncompleta + (promedios[1][3][jj] - promedio)^2
		end
	end
	varianza = varianIncompleta/samples
	desviacion = math.sqrt(varianIncompleta/(samples*(samples-1)))
}

Varianza muestral $s^2$
\begin{align*}
	s^2 &= \frac{1}{n} \sum^n_{i=1}(x_i-\overline{x})^2\\
	s^2 &= \frac{\directlua{tex.print(varianIncompleta)}}{\directlua{tex.print(samples)}}\\
	s^2 &= \directlua{tex.print(varianza)}
\end{align*}

Desviación típica $\sigma$
\begin{align*}
	\sigma &= \sqrt{ \frac{\sum^n_{i=1}(x_i-\overline{x})^2}{n(n-1)} }\\
	\sigma &= \sqrt{ \frac{\directlua{tex.print(varianIncompleta)}}{\directlua{tex.print((samples-1)*samples)}} }\\
	\sigma &= \directlua{tex.print(desviacion)}
\end{align*}

Incertidumbre
\begin{align*}
	(\directlua{tex.print(promedio)}\pm\directlua{tex.print(desviacion*10)}*10^{-1}) \frac{rad}{s^2}
\end{align*}

Esta invertido.
Y el valor absoluto medido es menor que el teórico.
En la prueba de la fórmula la masa $m_2$ estaba a la izquierda, mientras que en el laboratorio estaba a la derecha.

\subsection{Utilizando  TRABAJO  Y  ENERGÍA,
determinar  la  velocidad  final  de  la  masa  2 (m2).
Esto quiere decir, un instante antes de detenerse la masa 1.}%

Distancia del carrito a la pared $d=0.4m$
\begin{align*}
	W &= d_2*\cancel{m_2}g*\cancel{\cos{0}} = \frac{1}{2} \cancel{m_2}v^2\\
	v_2 &= \sqrt{2d_2g} && d_2 =\frac{2d}{r_1}*r_2\\
	v_2 &= \sqrt{2\frac{0.8\cancel{m}}{0.024\cancel{m}}*0.0145m*9.81 \frac{m}{s^2} }\\
	v_2 &= 3.08\frac{m}{s^2}
\end{align*}

\subsection{A partir de las mediciones realizadas,
inferir el valor de la velocidad final de la masa 2.
Compararlo con el valor teórico.
Comentar.}%

\shorthandoff{"}
%d=40cm{{{
\begin{figure}[H]
	\centering
	\begin{tikzpicture}
		% Opciones {{{
		[
			spy using outlines={rectangle,
				magnification=2,
				width=7.5cm,
				height=3cm,
				connect spies
			}
		]%}}}
		\begin{axis}[
			% Opciones {{{
			width=0.9\linewidth,
			height=6cm,
			title=\textbf{$v_2,\ d=40cm$},
			axis y line=left,
			axis x line=bottom,
			axis line style = ultra thick,
			xlabel={Tiempo ($s$)},
			ylabel={Velocidad bloque 2 ($\frac{m}{s}$)},
			ymajorgrids=true
			]%}}}
			\addplot[
				% Opciones {{{
				restrict x to domain=0:4,
				draw opacity=0.1,
				color=red,
				smooth,
				ultra thick
				]%}}}
				table
				[
					col sep=comma,
					%y index=2,
					y expr=\thisrowno{2}*0.0145,
					x index=0
				]
				{d3.csv};
			\addplot[
				% Opciones {{{
				restrict x to domain=0.6:1.75,
				color=red,
				smooth,
				ultra thick
				]%}}}
				table
				[
					col sep=comma,
					%y index=2,
					y expr=\thisrowno{2}*0.0145,
					x index=0
				]
				{d3.csv};
			\addplot[
				% Opciones {{{
				restrict x to domain=1.71:1.75,
				nodes near coords={
				$(\pgfmathprintnumber{\pgfkeysvalueof{/data point/x}}s,
				\pgfmathprintnumber{\pgfkeysvalueof{/data point/y}}\frac{m}{s^2})$},
				nodes near coords align={above},
				color=black,
				smooth,
				ultra thick
				]%}}}
				table
				[
					col sep=comma,
					%y index=2,
					y expr=\thisrowno{2}*0.0145,
					x index=0
				]
				{d3.csv};
			\coordinate (ripple) at (1.75,-0.72);
			\coordinate (lupa) at (3,0);
		\end{axis}
		\spy[black] on (ripple) in node[fill=white] at (lupa);
	\end{tikzpicture}
\end{figure}%}}}
\shorthandon{"}

Me siento confundido.
Es muy diferente al valor teórico.
Además está invertido.

\end{document}
